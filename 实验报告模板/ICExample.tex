%!xelatex = 'xelatex --halt-on-error %O %S'
\documentclass{iceli}
\begin{document}
\emptitle{实验名称}
\empauthor{作者}

% 奇数页页眉 % 请在这里写出第一作者以及论文题目
\fancyhead[CO]{{\footnotesize 作者: 实验题目}}
%--------------------------------------------------------------
\Keyword{关键词, 很关键的词, 十分关键的词, 有一些关键的词, 大关键词}
\twocolumn[
\begin{@twocolumnfalse}
\maketitle
\begin{empAbstract}
摘要内容
\end{empAbstract}
%%%%%%%%首页角注,依次为实验时间、报告时间、学号、email
\empfirstfoot{2022-02-25}{2022-2-26}{这里填学号}{这里填入邮件}
\end{@twocolumnfalse}
]
%%%%%%%%!首页角注可能与正文重叠,请通过调整正文中第一页的\enlargethispage{-3.3cm}位置手动校准正文底部位置:
%-------------------------------------------------------------
%  正文由此开始
\wuhao 
%  分栏开始
\enlargethispage{-1.5cm}%该命令可以尝试放在不同的地方和更改其中的参数是左下角的标注显示正确
\section{引~~言}
\subsection{实验背景}

\subsection{实验目的}
实验具体的内容,这里可能会用到列表结构
\begin{enumerate}
\renewcommand{\labelenumi}{(\theenumi)}
\item 目的一
\item 目的二
\end{enumerate} 

\cite{ceshi}.%这里的文献引入仅用于测试

\section{实验内容与数据处理}
\subsection{实验原理}
正文内容:正文文字五号宋体。

\subsection{实验内容}

\subsection{实验方法和技术}

\subsection{实验结果的分析和结论}
正文内容正文文字五号宋体。
\subsection{报告中公式、字母的规范写法}
公式全文统一编号,如公式为
\begin{equation}\label{EQ1}
\frac{\partial u}{\partial x}+\frac{\partial v}{\partial y}+\frac{\partial w}{\partial z}=0
\end{equation}
式中,$u$是××××(单位);$v$是×××(单位);$w$是××(单位)。

对于公式,应全文统一连续编号,如式\eqref{EQ1}……一般情况下,需要引用的或重要的公式才编号。在文中引用时,用“式(编号)”表示。
后文不再提及的,可以不编号。如
\begin{equation*}
1 + 1 + 3 = 5
\end{equation*}

对于公式中首次出现的量的符号,按照其在式中出现的顺序,用准确、简洁的语句对其进行逐一解释。公式中变量应尽量避免复合上下角标的使用;尽量少用3层关系的上下标,同时应尽量减少不必要的公式推导。

\subsection{实验遇到的问题及解决的方法}

\section{实验小结}
\subsection{体会与收获}

\subsection{实验建议}

%-------------------------------------------------------------
%  参考文献
%  参考文献按GB/T 7714-2015《文后参考文献著录规则》的要求著录. 
%  参考文献在正文中的引用方法:\cite{bib文件条目的第一行}

\renewcommand\refname{\heiti\wuhao\centerline{参考文献}\global\def\refname{参考文献}}
\vskip 12pt

\let\OLDthebibliography\thebibliography
\renewcommand\thebibliography[1]{
  \OLDthebibliography{#1}
  \setlength{\parskip}{0pt}
  \setlength{\itemsep}{0pt plus 0.3ex}
}

{
\renewcommand{\baselinestretch}{0.9}
\liuhao
\bibliographystyle{gbt7714-numerical}
\bibliography{./ICExample}
}

\end{document}


